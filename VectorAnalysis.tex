\documentclass[a4paper,11pt]{article}

% !TEX root=VectorAnalysis.tex

\usepackage[utf8]{inputenc}
\usepackage[T1]{fontenc} % FRO: macht Umlaute verfügbar (ö statt \"o, ...)
\usepackage{siunitx}
%\usepackage{amsmath, wasysym}
\usepackage[intlimits]{amsmath}
\usepackage{wasysym}
\usepackage[english]{babel}
\usepackage{amssymb}
\usepackage{amsfonts}
\usepackage{graphicx}
\usepackage[usenames,dvipsnames]{xcolor}
\usepackage{enumitem}
\usepackage{framed}

\newcommand{\be}{\begin{equation}}
\newcommand{\ee}{\end{equation}}

\newcommand{\vecvar}[1]{\boldsymbol{#1}}
\newcommand{\eng}[1]{\textcolor{OliveGreen}{\textit{#1}}}
\newcommand{\engb}[1]{(\textcolor{OliveGreen}{\textit{#1}})}

\newcommand{\eps}{\varepsilon}
\newcommand{\reyn}{\mathrm{Re}}
\newcommand{\keps}{$k$-$\eps$}

\newcommand{\bs}[1]{\boldsymbol{#1}}
\newcommand{\xvec}{\boldsymbol{x}}
\newcommand{\uvec}{\boldsymbol{u}}
\newcommand{\fvec}{\boldsymbol{f}}

\newcommand{\vvec}{\bs{v}}
\newcommand{\Fvec}{\bs{F}}
\newcommand{\Mvec}{\bs{M}}

\setlength{\oddsidemargin}{0.46cm}
\setlength{\evensidemargin}{0.46cm}
\setlength{\textwidth}{15cm}

\newcounter{ProblemCounter}
\newtheorem{example}{Example}

\newcommand{\problem}{\stepcounter{ProblemCounter}\vspace{1em}\noindent\textbf{Question \arabic{ProblemCounter}:} }
\newcommand{\solution}{\vspace{1em}\noindent\textbf{Solution}\\[1em]}



\newcommand{\mybox}[3]%
{\begin{center}\fbox{\begin{minipage}[c][#1][c]{#2} #3 \end{minipage}}\end{center}}







\begin{document}

\selectlanguage{english}

\title{Vector Analysis and Tensors}
\author{Henrik Nordborg}
%\date{Version 2017-02-05}
\maketitle

\section{Introduction}

\section{Functions of many variables and partial derivatives}


\subsection{Scalar and vector fields}

Continuum mechanics works with \emph{scalar} and \emph{vector fields}. A scalar field is a scalar function of multiple variables such as 
\be
\phi(x,y,z,t) = \phi(\xvec,t)
\ee

\subsection{Partial derivatives}

The partial derivaties are derived according to
\be
\frac{\partial \phi}{\partial x} = \lim_{h\rightarrow 0} \frac{\phi(x+h,y,z,t) - \phi(x,y,z,t)}{h}
\ee
with analogous defintions for ${\partial \phi}/{\partial y}$ and ${\partial \phi}/{\partial z}$.

Note that we we use a special symbol, the curved $\partial$, to differentiate the partial derivative from the total derivative. The reason why this is necessary can be understood by considering the following example. Consider a function of time and space $f(t,x)$. Let us assume that the  $x$-coordinate is a function of $t$. In this case, our function only depends on $t$ and we have
\be
g(t) = f(t,x(t)).
\ee
If we compute the derivate we obtain
\be
\frac{d g}{dt} = \frac{\partial f}{\partial t} + \frac{\partial f}{\partial x} \frac{dx}{dt}.
\ee
There are different notations for the partial deriative. One often finds
\be
\frac{\partial f}{\partial t}  = \partial_t f = f_t
\ee

\subsection{Gradient}

Consider a scalar field $\phi(x,y,z)$. The gradient of $\phi$ is a vector field with the components
\be
\mathrm{grad}\, \phi = \nabla \phi = \left( \frac{\partial \phi}{\partial x}, \frac{\partial \phi}{\partial y}, \frac{\partial \phi}{\partial z} \right).
\ee
The importance of the gradient can be understood by considering the change in the value of $\phi(x,y,z)$ under a small change of the arguments. We find
\be
\phi(x+\delta x ,y+\delta y,z+\delta z) \approx \phi(x,y,z) + \frac{\partial \phi}{\partial x} \delta x +  \frac{\partial \phi}{\partial y} \delta y + \frac{\partial \phi}{\partial z} \delta z
\ee
or
\be
\phi( \xvec + \delta \xvec ) = \phi( \xvec ) + \nabla\phi\cdot \delta\xvec + \mathcal{O}(|\delta\xvec|^2).
\ee
The gradient is therefore is the generalization of the derivative to many dimensions. The vector $\nabla \phi$ gives the direction of maximum increase of $\phi$. A local maximum or minimum is characterized by $\nabla \phi  = 0$.

\subsection{Divergence}

Consider a vector field 








\end{document}